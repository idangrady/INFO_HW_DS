\section{Pannenkoeken}


\begin{lstlisting}[caption=Insert Sort variant, escapechar=|]
A = new Array;  |\label{line:first}|
sign = Sign(A[1] - A[0]);
for ( i = 2; i < A.Length; i++) |\label{line:for}|
{
    // Skip any next number that is already in a sorted order
    new_sign = Sign(A[i] - A[i - 1]); |\label{line:six}|
    if (new_sign != 0 && new_sign != sign) |\label{line:comp-sign}|
    {
        // A incorrect order has been found.
        if (sign == 1 && (A[i] <= A[0] || A[i] >= A[i-1]) || sign == -1 && (A[i] >= A[0] || A[i] <= A[i - 1])) |\label{line:comp}|
        {
            // If number is a new highest or lowest, only one flip is required. 
            // In the worst case, the first two numbers A[0] and A[1] are the lowest and highest in the A.
            A = Flip(A, i);
            total_flips++;
            sign = new_sign;
        } else {
            // Binary search
            left = 0;
            right = A.Length - 1;
            reversed = (A[0] - A[A.Length - 1] > 0);
            while (right > left + 1)
            {
                // Bit shift is the same as deviding by 2. Saves a little time
                int mid = (left + right) >> 1;

                // Check if value is left or right.
                if (A[i] < arr[mid])
                {
                    if (reversed) { left = mid; } else { right = mid; } 
                } else {
                    if (reversed) { right = mid; } else { left = mid; }
                }
            }
            // Get position of the number closest to A[i] between 0 and i. 
            pos = left; 

            // Place item 'i' at pos.
            A = Flip(A, i + 1);
            A = Flip(A, i - pos + 1);
            A = Flip(A, i - pos);

            sign = new_sign;
            total_flips += 3;
        }
    }
}
\end{lstlisting}

In the best case, the array is already sorted and the comparisons would be $O(n)$ to check at line \ref{line:comp-sign} if any number is not ascending or descending as the previous two numbers. In this case the Flips would be 0.

In the worst case, the array is unsorted and the first two values are the min and max bound. Meaning the comparison on line \ref{line:comp} will always be false. In this case, the total flips would be $3*n$, since Flip() is called three times.

If we assume that a $Flip()$ takes $O(1)$ time, then we can deduce that the time complexity from this algorithm would be O(n \log(n)). 
Normally Insertion Sort would be $O(n^2)$ because of the swaps, but in this case no iterative swaps are executed, only 3 $Flip()$'s which are $O(1)$.

Lets assume that each operation or comparison takes Ci time. Each row of the algorithm above has a number, counting down in descending order, so line \ref{line:first} would be C1, the for-loop at line \ref{line:for} would be C3a, C3b and C3c and at line \ref{line:six} would be C6.

The time it takes to run this algorithm in the worst case would be:

\begin{align*}
C1 + C2 + C3a + (n+1)*C3b + \\
n(C3c + C6 + C7 + C10 + C19 + C20 + C21 + \\
2\log(n) * (C22 + C25 + C28 + 0.5*C30 + 0.5*C32) + \\
C36 + C39 + C40 + C41 + C43 + C44)
\end{align*}

Total comparisons will be  $n* \log_{2}(n)$

This is equal to $O(n*\log(n))$
Worst total flips would be $n*C39 + n*C40 + n*C41$ which would be $3*n$


\vspace{15mm}

\begin{lstlisting}[caption=Selection Sort variant, escapechar=|]
for (int i = 0; i < input.Length; i++) |\label{line2:for}|
{
    // Array 0 to i has already been sorted
    int lowest = i;
    int lowest_val = input[i];
    for(int j = i; j<input.Length; j++)
    {
        if (input[j] < lowest_val)
        {
            lowest = j;
            lowest_val = input[j];
        }
    }

    input = flipIntArrayAt(input, lowest);
    input = flipIntArrayAt(input, lowest + 1);

    total_flips += 2;
}
\end{lstlisting}

Vanwege de for-loop op lijn \ref{line2:for} zal het aantal flips hier altijd $2*n$ zijn.
Maar er zit een doubele for-loop in, dus als we aannemen dat $Flip()$ een tijd van $O(1)$ heeft, 
dan zal het algoritme $O(n^2)$ snel zijn. 


\vspace{15mm}

\begin{lstlisting}[caption=Quick Sort Variant, escapechar=|]
pannenkoekenQS(A, low, high)
{
    total_flips = 0;
    if (low < high)
    {
        pivot_pos, added_flips;
        (A, pivot_pos, added_flips) = pannenkoekenQSPart(A, low, high);
        total_flips += added_flips;

        (A, added_flips) = pannenkoekenQS(A, low, pivot_pos - 1);
        total_flips += added_flips;
        (A, added_flips) = pannenkoekenQS(A, pivot_pos + 1, high);
        total_flips += added_flips;
    }

    return (A, total_flips);
}

pannenkoekenQSPart(A,  low,  high) |\label{line3:pannenkoekenQSPart}|
{
     pivot = A[low];
     left_wall = low;
     total_flips = 0;
     add_flips;

    for ( i = low + 1; i <= high; i++) 
    {
        if (A[i] < pivot) {
            (A, add_flips) = pannenkoekenSwap(A, i, left_wall + 1);
            left_wall++;
            total_flips += add_flips;
        }
    }

    (A, add_flips) = pannenkoekenSwap(A, low, left_wall);
    total_flips += add_flips;

    return (A, left_wall, total_flips);
}

pannenkoekenSwap(A,  pos1,  pos2)
{
    if (pos1 == pos2)
    {
        return (A, 0);
    }
    else
    {
        // Get left and right position
        left = (pos1 < pos2) ? pos1 : pos2;
        right = (pos1 < pos2) ? pos2 : pos1;

        // Swap two positions
        A = Flip(A, left + 1);
        A = Flip(A, right + 1);
        A = Flip(A, right);
        A = Flip(A, right - 1);
        A = Flip(A, right);
        A = Flip(A, left + 1);

        return (A, 6);
    }
}
\end{lstlisting}

De hoeveelheid keer de functie op lijn \ref{line3:pannenkoekenQSPart} wordt uitgeroepen is ongeveer everedig met:

\[2*\log_{\frac{4}{3}}(n)\]

Want als we vanuit gaan dat de split gebeurt bij een kwart van elk eind van de array een slechte split is, 
zal $50\%$ van de tijd een goede split zijn waarbij na de split er een deel over blijft met minder of gelijk aan $3/4$ van het gesplitte array.

$A_n$ is een array n keer gesplit. 
Een snelle split: $A_i <= 3/4 * (A_i_-_1)$
Dus voor $Q$ snelle splits geldt voor de array grote $<= n*(3/4)^Q$

Het splitten stopt als $n*(3/4)^Q <= 1$

\begin{align*}
n*(3/4)^Q <= 1  \\
\Rightarrow \log_{\frac{3}{4}}(1/n) >= Q \\
\Rightarrow \log_{\frac{4}{3}}(n) >= Q
\end{align*}

Maar elke split heeft maar een $50\%$ kans om goed te zijn, dus de hoeveelheid splits nodig is:

\begin{align*}
\log_{\frac{4}{3}}(n) >= 0.5 / Q \\
\Rightarrow 2 * \log_{\frac{4}{3}}(n) >= Q
\end{align*}

Dus de vergelijkingen per key is $2*\log_{\frac{4}{3}}(n)$.
Dat geeft ons: $2n * \log_{\frac{4}{3}}(n)$ wat correspondeerd met $O(n*\log(n))$


De formule $2n * \log_{\frac{4}{3}}(n)$ voor waardes van $n > 1$ is altijd kleiner dan $n * \log_{2}(n)$ van Insert Sort, dus als $Flip()$ een $O(1)$ is, dan is Quick Sort beter.